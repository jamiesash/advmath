% Options for packages loaded elsewhere
\PassOptionsToPackage{unicode}{hyperref}
\PassOptionsToPackage{hyphens}{url}
%
\documentclass[
]{article}
\usepackage{amsmath,amssymb}
\usepackage{lmodern}
\usepackage{iftex}
\ifPDFTeX
  \usepackage[T1]{fontenc}
  \usepackage[utf8]{inputenc}
  \usepackage{textcomp} % provide euro and other symbols
\else % if luatex or xetex
  \usepackage{unicode-math}
  \defaultfontfeatures{Scale=MatchLowercase}
  \defaultfontfeatures[\rmfamily]{Ligatures=TeX,Scale=1}
\fi
% Use upquote if available, for straight quotes in verbatim environments
\IfFileExists{upquote.sty}{\usepackage{upquote}}{}
\IfFileExists{microtype.sty}{% use microtype if available
  \usepackage[]{microtype}
  \UseMicrotypeSet[protrusion]{basicmath} % disable protrusion for tt fonts
}{}
\makeatletter
\@ifundefined{KOMAClassName}{% if non-KOMA class
  \IfFileExists{parskip.sty}{%
    \usepackage{parskip}
  }{% else
    \setlength{\parindent}{0pt}
    \setlength{\parskip}{6pt plus 2pt minus 1pt}}
}{% if KOMA class
  \KOMAoptions{parskip=half}}
\makeatother
\usepackage{xcolor}
\usepackage[margin=1in]{geometry}
\usepackage{graphicx}
\makeatletter
\def\maxwidth{\ifdim\Gin@nat@width>\linewidth\linewidth\else\Gin@nat@width\fi}
\def\maxheight{\ifdim\Gin@nat@height>\textheight\textheight\else\Gin@nat@height\fi}
\makeatother
% Scale images if necessary, so that they will not overflow the page
% margins by default, and it is still possible to overwrite the defaults
% using explicit options in \includegraphics[width, height, ...]{}
\setkeys{Gin}{width=\maxwidth,height=\maxheight,keepaspectratio}
% Set default figure placement to htbp
\makeatletter
\def\fps@figure{htbp}
\makeatother
\setlength{\emergencystretch}{3em} % prevent overfull lines
\providecommand{\tightlist}{%
  \setlength{\itemsep}{0pt}\setlength{\parskip}{0pt}}
\setcounter{secnumdepth}{-\maxdimen} % remove section numbering
\ifLuaTeX
  \usepackage{selnolig}  % disable illegal ligatures
\fi
\IfFileExists{bookmark.sty}{\usepackage{bookmark}}{\usepackage{hyperref}}
\IfFileExists{xurl.sty}{\usepackage{xurl}}{} % add URL line breaks if available
\urlstyle{same} % disable monospaced font for URLs
\hypersetup{
  pdftitle={Homework: Set proofs},
  hidelinks,
  pdfcreator={LaTeX via pandoc}}

\title{Homework: Set proofs}
\author{}
\date{\vspace{-2.5em}2022-10-22}

\begin{document}
\maketitle

\hypertarget{part-1}{%
\section{Part 1:}\label{part-1}}

\hypertarget{part-2-set-proofs}{%
\section{Part 2: Set Proofs}\label{part-2-set-proofs}}

\textbf{Sets and power sets:}

Question 2: Remember that for a set \(A\), the power set of \(A\) is the
set of all subsets of \(A\). We write \(P(A)\).

\begin{enumerate}
\def\labelenumi{(\alph{enumi})}
\tightlist
\item
  Prove that if \(A \subseteq B\) then \(P(A) \subseteq P(B)\).
\item
  Prove that in general \(P(A) \cup P(B) \neq P(A \cup B)\).
\item
  Can you find sets \(A\) and \(B\) where it is true that
  \(P(A) \cup P(B) = P(A\cup B)\)?
\item
  Prove or disprove: \(P(A) \cap P(B) = P(A \cap B)\) for all sets \(A\)
  and \(B\).
\end{enumerate}

\textbf{Computation:}

Question 4: Find all values of \(n\) that make the congruence true. (a)
\(5 \equiv 25 (\mod n)\) (b) \(20 \equiv 0 (\mod n)\) (c)
\(37 \equiv 1 (\mod n)\)

\textbf{Proofs:}

Question 5: If n is odd, then \(8|(n^2-1)\).

\emph{Proof} We proceed by inference and cases.

By the definition of divisibility, \(8|(n^2-1)\) implies \(8l = n^2-1\)
where \(l \in \mathbb{Z}\). By the definition of odd numbers, we have n
is \(2k +1 \in Z\). We proceed with two cases, first where n is any
positive odd number, and then where n is any negative odd number.

(Base step) Consider the series of positive odd numbers, \(H\),
(\(k \in N ,0:2k+1\)) that is ordered from smallest to largest (not sure
how to use series/set notation to denote order). Now consider
\(m \in H\), where first possible value of \(m\) is 1, coresponding to
the first possible value of k being 0. Through substitution we
get\ldots{}

\[
8l = m^2-1 \\
8l = 1^2-1 \\
8l = 0 \\
l = 0 \\
or \\
8l = (2k +1)^2-1 \\
8l = (2(0)+1)^2-1 \\
8l = (1)^2-1 \\
8l = 0 \\
l = 0
\] (intuitive step) Suppose \(8|(n^2-1)\) where \(n\) is any positive
odd number.\\
By the definition of divisibiklity and odd numbers we have\ldots{} \[
\begin{equation} 
\begin{split}
8l = (2k +1)^2-1 & \text{ definition of divisibility } \\
8l + (8k+8) = (2k +1)^2-1 + (8k+8) & \text{ add (8k+8) to both sides} \\
8l + 8k+8 = (2k + 3)^2-1 & \text{ factor right side} \\
8(l + k + 1) = (2k + 3)^2-1 & \text{ pull out the 8} \\
8p = (2k + 3)^2-1 & \text{ addition of intigers l + k + 1 = p where } p \in \mathbb{Z} \\
\end{split}
\end{equation} 
\]

therefore \(8|(2l+3)^2-1\) or \(8|(n+2)^2-1\) i.e.~\(S_{k+1}\). So
\(S_k \implies S_{k+1}\).

\hypertarget{part-3-reflection}{%
\section{Part 3: Reflection}\label{part-3-reflection}}

\end{document}
